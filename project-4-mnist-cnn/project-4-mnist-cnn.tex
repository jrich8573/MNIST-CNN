%% template.tex
%% from
%% bare_conf.tex
%% V1.4b
%% 2015/08/26
%% by Michael Shell
%% See:
%% http://www.michaelshell.org/
%% for current contact information.
%%
%% This is a skeleton file demonstrating the use of IEEEtran.cls
%% (requires IEEEtran.cls version 1.8b or later) with an IEEE
%% conference paper.
%%
%% Support sites:
%% http://www.michaelshell.org/tex/ieeetran/
%% http://www.ctan.org/pkg/ieeetran
%% and
%% http://www.ieee.org/

%%*************************************************************************
%% Legal Notice:
%% This code is offered as-is without any warranty either expressed or
%% implied; without even the implied warranty of MERCHANTABILITY or
%% FITNESS FOR A PARTICULAR PURPOSE!
%% User assumes all risk.
%% In no event shall the IEEE or any contributor to this code be liable for
%% any damages or losses, including, but not limited to, incidental,
%% consequential, or any other damages, resulting from the use or misuse
%% of any information contained here.
%%
%% All comments are the opinions of their respective authors and are not
%% necessarily endorsed by the IEEE.
%%
%% This work is distributed under the LaTeX Project Public License (LPPL)
%% ( http://www.latex-project.org/ ) version 1.3, and may be freely used,
%% distributed and modified. A copy of the LPPL, version 1.3, is included
%% in the base LaTeX documentation of all distributions of LaTeX released
%% 2003/12/01 or later.
%% Retain all contribution notices and credits.
%% ** Modified files should be clearly indicated as such, including  **
%% ** renaming them and changing author support contact information. **
%%*************************************************************************


% *** Authors should verify (and, if needed, correct) their LaTeX system  ***
% *** with the testflow diagnostic prior to trusting their LaTeX platform ***
% *** with production work. The IEEE's font choices and paper sizes can   ***
% *** trigger bugs that do not appear when using other class files.       ***                          ***
% The testflow support page is at:
% http://www.michaelshell.org/tex/testflow/

\documentclass[conference,final,]{IEEEtran}
% Some Computer Society conferences also require the compsoc mode option,
% but others use the standard conference format.
%
% If IEEEtran.cls has not been installed into the LaTeX system files,
% manually specify the path to it like:
% \documentclass[conference]{../sty/IEEEtran}





% Some very useful LaTeX packages include:
% (uncomment the ones you want to load)


% *** MISC UTILITY PACKAGES ***
%
%\usepackage{ifpdf}
% Heiko Oberdiek's ifpdf.sty is very useful if you need conditional
% compilation based on whether the output is pdf or dvi.
% usage:
% \ifpdf
%   % pdf code
% \else
%   % dvi code
% \fi
% The latest version of ifpdf.sty can be obtained from:
% http://www.ctan.org/pkg/ifpdf
% Also, note that IEEEtran.cls V1.7 and later provides a builtin
% \ifCLASSINFOpdf conditional that works the same way.
% When switching from latex to pdflatex and vice-versa, the compiler may
% have to be run twice to clear warning/error messages.






% *** CITATION PACKAGES ***
%
%\usepackage{cite}
% cite.sty was written by Donald Arseneau
% V1.6 and later of IEEEtran pre-defines the format of the cite.sty package
% \cite{} output to follow that of the IEEE. Loading the cite package will
% result in citation numbers being automatically sorted and properly
% "compressed/ranged". e.g., [1], [9], [2], [7], [5], [6] without using
% cite.sty will become [1], [2], [5]--[7], [9] using cite.sty. cite.sty's
% \cite will automatically add leading space, if needed. Use cite.sty's
% noadjust option (cite.sty V3.8 and later) if you want to turn this off
% such as if a citation ever needs to be enclosed in parenthesis.
% cite.sty is already installed on most LaTeX systems. Be sure and use
% version 5.0 (2009-03-20) and later if using hyperref.sty.
% The latest version can be obtained at:
% http://www.ctan.org/pkg/cite
% The documentation is contained in the cite.sty file itself.






% *** GRAPHICS RELATED PACKAGES ***
%
\ifCLASSINFOpdf
  % \usepackage[pdftex]{graphicx}
  % declare the path(s) where your graphic files are
  % \graphicspath{{../pdf/}{../jpeg/}}
  % and their extensions so you won't have to specify these with
  % every instance of \includegraphics
  % \DeclareGraphicsExtensions{.pdf,.jpeg,.png}
\else
  % or other class option (dvipsone, dvipdf, if not using dvips). graphicx
  % will default to the driver specified in the system graphics.cfg if no
  % driver is specified.
  % \usepackage[dvips]{graphicx}
  % declare the path(s) where your graphic files are
  % \graphicspath{{../eps/}}
  % and their extensions so you won't have to specify these with
  % every instance of \includegraphics
  % \DeclareGraphicsExtensions{.eps}
\fi
% graphicx was written by David Carlisle and Sebastian Rahtz. It is
% required if you want graphics, photos, etc. graphicx.sty is already
% installed on most LaTeX systems. The latest version and documentation
% can be obtained at:
% http://www.ctan.org/pkg/graphicx
% Another good source of documentation is "Using Imported Graphics in
% LaTeX2e" by Keith Reckdahl which can be found at:
% http://www.ctan.org/pkg/epslatex
%
% latex, and pdflatex in dvi mode, support graphics in encapsulated
% postscript (.eps) format. pdflatex in pdf mode supports graphics
% in .pdf, .jpeg, .png and .mps (metapost) formats. Users should ensure
% that all non-photo figures use a vector format (.eps, .pdf, .mps) and
% not a bitmapped formats (.jpeg, .png). The IEEE frowns on bitmapped formats
% which can result in "jaggedy"/blurry rendering of lines and letters as
% well as large increases in file sizes.
%
% You can find documentation about the pdfTeX application at:
% http://www.tug.org/applications/pdftex





% *** MATH PACKAGES ***
%
%\usepackage{amsmath}
% A popular package from the American Mathematical Society that provides
% many useful and powerful commands for dealing with mathematics.
%
% Note that the amsmath package sets \interdisplaylinepenalty to 10000
% thus preventing page breaks from occurring within multiline equations. Use:
%\interdisplaylinepenalty=2500
% after loading amsmath to restore such page breaks as IEEEtran.cls normally
% does. amsmath.sty is already installed on most LaTeX systems. The latest
% version and documentation can be obtained at:
% http://www.ctan.org/pkg/amsmath





% *** SPECIALIZED LIST PACKAGES ***
%
%\usepackage{algorithmic}
% algorithmic.sty was written by Peter Williams and Rogerio Brito.
% This package provides an algorithmic environment fo describing algorithms.
% You can use the algorithmic environment in-text or within a figure
% environment to provide for a floating algorithm. Do NOT use the algorithm
% floating environment provided by algorithm.sty (by the same authors) or
% algorithm2e.sty (by Christophe Fiorio) as the IEEE does not use dedicated
% algorithm float types and packages that provide these will not provide
% correct IEEE style captions. The latest version and documentation of
% algorithmic.sty can be obtained at:
% http://www.ctan.org/pkg/algorithms
% Also of interest may be the (relatively newer and more customizable)
% algorithmicx.sty package by Szasz Janos:
% http://www.ctan.org/pkg/algorithmicx




% *** ALIGNMENT PACKAGES ***
%
%\usepackage{array}
% Frank Mittelbach's and David Carlisle's array.sty patches and improves
% the standard LaTeX2e array and tabular environments to provide better
% appearance and additional user controls. As the default LaTeX2e table
% generation code is lacking to the point of almost being broken with
% respect to the quality of the end results, all users are strongly
% advised to use an enhanced (at the very least that provided by array.sty)
% set of table tools. array.sty is already installed on most systems. The
% latest version and documentation can be obtained at:
% http://www.ctan.org/pkg/array


% IEEEtran contains the IEEEeqnarray family of commands that can be used to
% generate multiline equations as well as matrices, tables, etc., of high
% quality.




% *** SUBFIGURE PACKAGES ***
%\ifCLASSOPTIONcompsoc
%  \usepackage[caption=false,font=normalsize,labelfont=sf,textfont=sf]{subfig}
%\else
%  \usepackage[caption=false,font=footnotesize]{subfig}
%\fi
% subfig.sty, written by Steven Douglas Cochran, is the modern replacement
% for subfigure.sty, the latter of which is no longer maintained and is
% incompatible with some LaTeX packages including fixltx2e. However,
% subfig.sty requires and automatically loads Axel Sommerfeldt's caption.sty
% which will override IEEEtran.cls' handling of captions and this will result
% in non-IEEE style figure/table captions. To prevent this problem, be sure
% and invoke subfig.sty's "caption=false" package option (available since
% subfig.sty version 1.3, 2005/06/28) as this is will preserve IEEEtran.cls
% handling of captions.
% Note that the Computer Society format requires a larger sans serif font
% than the serif footnote size font used in traditional IEEE formatting
% and thus the need to invoke different subfig.sty package options depending
% on whether compsoc mode has been enabled.
%
% The latest version and documentation of subfig.sty can be obtained at:
% http://www.ctan.org/pkg/subfig




% *** FLOAT PACKAGES ***
%

%\usepackage{fixltx2e}
% fixltx2e, the successor to the earlier fix2col.sty, was written by
% Frank Mittelbach and David Carlisle. This package corrects a few problems
% in the LaTeX2e kernel, the most notable of which is that in current
% LaTeX2e releases, the ordering of single and double column floats is not
% guaranteed to be preserved. Thus, an unpatched LaTeX2e can allow a
% single column figure to be placed prior to an earlier double column
% figure.
% Be aware that LaTeX2e kernels dated 2015 and later have fixltx2e.sty's
% corrections already built into the system in which case a warning will
% be issued if an attempt is made to load fixltx2e.sty as it is no longer
% needed.
% The latest version and documentation can be found at:
% http://www.ctan.org/pkg/fixltx2e


%\usepackage{stfloats}
% stfloats.sty was written by Sigitas Tolusis. This package gives LaTeX2e
% the ability to do double column floats at the bottom of the page as well
% as the top. (e.g., "\begin{figure*}[!b]" is not normally possible in
% LaTeX2e). It also provides a command:
%\fnbelowfloat
% to enable the placement of footnotes below bottom floats (the standard
% LaTeX2e kernel puts them above bottom floats). This is an invasive package
% which rewrites many portions of the LaTeX2e float routines. It may not work
% with other packages that modify the LaTeX2e float routines. The latest
% version and documentation can be obtained at:
% http://www.ctan.org/pkg/stfloats
% Do not use the stfloats baselinefloat ability as the IEEE does not allow
% \baselineskip to stretch. Authors submitting work to the IEEE should note
% that the IEEE rarely uses double column equations and that authors should try
% to avoid such use. Do not be tempted to use the cuted.sty or midfloat.sty
% packages (also by Sigitas Tolusis) as the IEEE does not format its papers in
% such ways.
% Do not attempt to use stfloats with fixltx2e as they are incompatible.
% Instead, use Morten Hogholm'a dblfloatfix which combines the features
% of both fixltx2e and stfloats:
%
% \usepackage{dblfloatfix}
% The latest version can be found at:
% http://www.ctan.org/pkg/dblfloatfix




% *** PDF, URL AND HYPERLINK PACKAGES ***
%
%\usepackage{url}
% url.sty was written by Donald Arseneau. It provides better support for
% handling and breaking URLs. url.sty is already installed on most LaTeX
% systems. The latest version and documentation can be obtained at:
% http://www.ctan.org/pkg/url
% Basically, \url{my_url_here}.




% *** Do not adjust lengths that control margins, column widths, etc. ***
% *** Do not use packages that alter fonts (such as pslatex).         ***
% There should be no need to do such things with IEEEtran.cls V1.6 and later.
% (Unless specifically asked to do so by the journal or conference you plan
% to submit to, of course. )



%% BEGIN MY ADDITIONS %%



\usepackage[unicode=true]{hyperref}

\hypersetup{
            pdftitle={MNIST Image Classification with Convolutional Neural Network},
            pdfborder={0 0 0},
            breaklinks=true}
\urlstyle{same}  % don't use monospace font for urls

% Pandoc toggle for numbering sections (defaults to be off)
\setcounter{secnumdepth}{0}
% Pandoc header
\usepackage{cite}

\providecommand{\tightlist}{%
  \setlength{\itemsep}{0pt}\setlength{\parskip}{0pt}}

%% END MY ADDITIONS %%


\hyphenation{op-tical net-works semi-conduc-tor}

\begin{document}
%
% paper title
% Titles are generally capitalized except for words such as a, an, and, as,
% at, but, by, for, in, nor, of, on, or, the, to and up, which are usually
% not capitalized unless they are the first or last word of the title.
% Linebreaks \\ can be used within to get better formatting as desired.
% Do not put math or special symbols in the title.
\title{MNIST Image Classification with Convolutional Neural Network}

% author names and affiliations
% use a multiple column layout for up to three different
% affiliations

\author{
\IEEEauthorblockN{Jason Rich}
\IEEEauthorblockA{Old Dominion University\\
Computer Science\\
Norfolk, Virginia 23504\\
jrich069@odu.edu
}
}

% conference papers do not typically use \thanks and this command
% is locked out in conference mode. If really needed, such as for
% the acknowledgment of grants, issue a \IEEEoverridecommandlockouts
% after \documentclass

% for over three affiliations, or if they all won't fit within the width
% of the page, use this alternative format:
%
%\author{\IEEEauthorblockN{Michael Shell\IEEEauthorrefmark{1},
%Homer Simpson\IEEEauthorrefmark{2},
%James Kirk\IEEEauthorrefmark{3},
%Montgomery Scott\IEEEauthorrefmark{3} and
%Eldon Tyrell\IEEEauthorrefmark{4}}
%\IEEEauthorblockA{\IEEEauthorrefmark{1}School of Electrical and Computer Engineering\\
%Georgia Institute of Technology,
%Atlanta, Georgia 30332--0250\\ Email: see http://www.michaelshell.org/contact.html}
%\IEEEauthorblockA{\IEEEauthorrefmark{2}Twentieth Century Fox, Springfield, USA\\
%Email: homer@thesimpsons.com}
%\IEEEauthorblockA{\IEEEauthorrefmark{3}Starfleet Academy, San Francisco, California 96678-2391\\
%Telephone: (800) 555--1212, Fax: (888) 555--1212}
%\IEEEauthorblockA{\IEEEauthorrefmark{4}Tyrell Inc., 123 Replicant Street, Los Angeles, California 90210--4321}}




% use for special paper notices
%\IEEEspecialpapernotice{(Invited Paper)}




% make the title area
\maketitle

% As a general rule, do not put math, special symbols or citations
% in the abstract
\begin{abstract}
In this article, I will demonstrate the use of a Convolutional Neural
Networks (CNN) as a technique for image classification. The dataset used
for this study is The MNIST database of handwritten digits, which
contains a training set of 60,000 examples, and a test set of 10,000
example. The dataset is a subset of the larger set available from
National Institute of Standards and Technology {[}1{]}. The goal of this
paper is to show that analyzing the MNIST data, using Anaconda's python
3.5 distribution, and Google's TensorFlow package for python3, on a
standard laptop is not only possible, but also efficient, accurate, and
certainly affordable. Moreover, I will show that CNN will converage in
as little as 2000 steps, and that as the steps increase, the error rate
draws closer and closer to zero, as the accuracy of the model grows
closer and closer to 100\%.
\end{abstract}

% no keywords

% use for special paper notices



% make the title area
\maketitle

% no keywords

% For peer review papers, you can put extra information on the cover
% page as needed:
% \ifCLASSOPTIONpeerreview
% \begin{center} \bfseries EDICS Category: 3-BBND \end{center}
% \fi
%
% For peerreview papers, this IEEEtran command inserts a page break and
% creates the second title. It will be ignored for other modes.
\IEEEpeerreviewmaketitle


\section{I. Introduction}\label{i.-introduction}

Historically, to preform image processing, whether high quality, digital
examples, or hand written notes, presented using a standard office
scanner, the machine learning practioner would have to extract language
dependent features like curvature of different letters, spacing, black
and white letter, etc., only to use a classifier such as Support Vector
Machine (SVM) to distingish between writers {[}2{]}. With the
publication of (LeCun et al. 1998), the analysis of handwritten,
variable, 2D shapes with Convolutional Neural Network was shown to
outpreform all other techniques {[}1{]}.

I will show that given the advance in Application Program Interface
frameworks, such as TensorFlow {[}3{]}, Keras {[}4{]}, H2O {[}5{]}
as-well-as others, have provided not only machine learning researchers
and practioners the ability and tools to quickly and efficiently analyze
larges amounts of data, with what are traditionally thought of as
mathematically complex, but also overly expensive, both runtime and
monetarily.

The key observation in this study was, given a well studied dataset, and
an evolving deep learning algorithm, the ability of personal hardware,
in my case my 2011 Mac Book Pro, with 16GB of RAM, a 1TB hardware, and
an i5 Intell processor, to reproduce results originally calculated on
academic or remote research servers. This says a lot about the hardware,
but more so about the work, research, and improvements that have rollup
into the current versions of modern day deep learning algorithms.

Hopefully, by the conclusion of this paper, I will have shown, that we
have come a long way the field of deep learning. However, I also hope to
show thar we have much more work remaining, and efforts in the fields of
quantum machine learning, quantum deep learning, and continued
improvment in high performance computing, are quintessential to furhter
the advancements, demonstrated within this paper.

\section{II. Related Work}\label{ii.-related-work}

\subsection{A. Foundational Work}\label{a.-foundational-work}

LeCun et al. (1998) laid the foundation ground work for all current
convolutional neural network architecture and image processing, building
on the concepts of Gradient-Based Learning. The work of LeCun et al.
(1998), and others, set the tone for work that is happening today.
Without the work of people like LeCun, Hinton, and Ng, we may not have
the bleeding edge algorithms or the tools to analyze the data we can
today.

\subsection{B. Gradient-Based
Learning}\label{b.-gradient-based-learning}

The general problem of minimizing a function with respect to a set of
parameter is at the root of many issues in computer science.
Gradient-Based Learning draws on the fact that it is generally much
easier to minimize a reasonably smooth, continuous fucntion than a
discrete (combinatorial) function. This is measured by the gradient of
the loss function with repect to the parameters. Efficient learning
algorithms can be devised when the gradient vector can be computed
analytically (as opposed to numerically through perturbation).
Furthermore, LeCun et al. (1998) notes; \ldots{}the basis of numerous
gradient-based learning algorithms with continuous-valued parameter. In
the procedure described continuous-values parameters \(W\) is a
real-valued vector, with respect to which \(E(W)\) is continuous, as
well as differentiable almost everywhere. {[}T{]}he simplest
minimization procedure in such a setting is ther gradient descent
algorithm where \(W\) is iteratively adjusted as follows:

\[W_k = W_{k-1}-\epsilon\frac{\partial \mathbf{E}(W)}{\partial W}\] In
the simplest case, \(\epsilon\) is a scalor constant {[}1{]}. Moreover,
LeCun et al. (1998) note: A poplar minimization procedure is the
stochastic gradient algorithm, also call the the on-line update. It
consists in updating the parameter vector using a noisy, or
approximated, version of the average gradient. In the most common
instance of it, \(W\) is updated on the basis of a single sample:
\[W_k = W_{k-1}-\epsilon\frac{\partial \mathbf{E}^{p_k}(W)}{\partial W}\]
With this procedure the parameter vector fluctutates around an average
trajectory, but usually converages considerably faster than a regular
gradient descent and second order methods on large training set with
redundant sample\ldots{}{[}1{]}. For more information on stochastic
gradient descent models see Bottou (2010) and Sutskever et al. (2013).

\subsection{C. Image Processing}\label{c.-image-processing}

However, with the advent of more sophisticated digital carmers, with
great pixel quality, and pixels pre-inch, images become larger and
larger. The traditional methods of image classification, using a
fully-connected network, with hundreds of hidden units in the first
layer {[}1{]}, {[}7{]}, {[}8{]}, creates thousands of weights.
Furthermore, using a fully-connected network negates the fact that
neightboring pixels are more coorelated that non-neighboring pixels
{[}7{]}.

The primary advantage of using a convolutional neural network is the
convolution itself. Convolutional neural networks are specifically
designed for processing data that has a know grid-like topology {[}8{]}.
Image data, as noted in Goodfellow, Bengio, and Courville (2016), should
be thought of as a 2-D grid of pixels. I will provide a brief summary of
convolution in section III, as well as the key differences in machine
learning and deep learning.

\section{III. Convultional Neural
Net}\label{iii.-convultional-neural-net}

\subsubsection{A. Convolution}\label{a.-convolution}

\[S(t)=\int x(a)w(t-a)da\] , annotated another way:\\
\[S(t)= (x*w)(t)\]

\subsubsection{B. Deep Learning}\label{b.-deep-learning}

\section{IV. Experiment}\label{iv.-experiment}

\subsection{A. Dataset}\label{a.-dataset}

The dataset used for the study in the MNIST {[}3{]}, extracted using
TensorFlow. The dataset used for this study, is a subset of a much
larger dataset, orignally made available by NIST {[}1{]}. It consist of
60,000 images for training the models, and 10,000 images for testing the
models.

The images in the dataset were pre-processed and stored as a greyscale,
centered \(28x28\) fixed-size image. The pre-processing performed on the
images, greatly improves the algorithms ability to process the data,
thus assisting in minimizing the error rate.

Other than the image files, the dataset also includes the label for
classifying the images. The values of the labels are on a range from
\(0\) to \(9\). The image training dataset is approximately \(0.099\)
gigabytes and the image testing dataset is considerably smaller.

The dataset was pulled locally using
the\texttt{tensorflow.examples.tutorials.mnist}module, and
calling\texttt{input\_data}funciotn with one hot encoding.

I will fully explain the code in the next subsection.

\subsection{B. Code}\label{b.-code}

\subsection{C. Results}\label{c.-results}

\section{V. Conclusion and Future
Work}\label{v.-conclusion-and-future-work}

The conclusion goes here.

\section{Acknowledgment}\label{acknowledgment}

The authors would like to thank\ldots{}

\newpage

\section*{References}\label{references}
\addcontentsline{toc}{section}{References}

\hypertarget{refs}{}
\hypertarget{ref-sgd2010}{}
Bottou, L. 2010. ``Large-Scale Machine Learning with Stochastic Gradient
Descent.'' In \emph{Proceedings of 19th International Conference
Computer Statistics}, 177--86. Princeton, NJ: Springer.

\hypertarget{ref-goodfellow2016}{}
Goodfellow, I, Y Bengio, and A Courville. 2016. \emph{Deep Learning}.
1st ed. Cambridge, MA: MIT Press. \url{http://www.deeplearningbook.org}.

\hypertarget{ref-lecun1998}{}
LeCun, Y, L Bottou, Y Bengio, and P Haffner. 1998. ``Gradient-Based
Learning Applied to Document Recognition.'' In \emph{Proceedings of the
IEEE}, 86:2278--2324. 11.
\url{http://yann.lecun.com/exdb/publis/pdf/cox-98.pdf}.

\hypertarget{ref-sgd2013}{}
Sutskever, I, J Martens, G Dahl, and G Hinton. 2013. ``On the Importance
of Initialization and Momentum in Deep Learning.'' In \emph{Proceedings
of the 30th International Conference on Machine Learning}, 1139--47.
ICML.

\end{document}


